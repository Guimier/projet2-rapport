\subsection{\apa}

Pour travailler sur l’interface de facturation web en PHP, nous avons installé {\apa} :

\todo[Vérifier le nom du paquet qui contient mod-php]
\begin{verbatim}
apt-get install apache2 libapache2-mod-php5
\end{verbatim}

La configuration par défaut d’{\apa} est suffisante pour notre travail : la racine web est définie comme le répertoire \texttt{/var/www}, dans lequel nous avons ajouté un répertoire \texttt{logs} qui contenait notre code. Nous n’avons pas travaillé directement à la racine afin de pouvoir utiliser {\pma} dans un autre répertoire.

Nous pouvions donc accéder à notre interface web \textit{via} l’adresse
\begin{verbatim}
http://localhost/logs/
\end{verbatim}
