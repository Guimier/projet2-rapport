\documentclass[a4paper, oneside, 12pt]{article}

%%%%%%%%%% Paquets %%%%%%%%%%

% Paquets pour le Français
\usepackage[utf8]{inputenc} % Gestion encodages
\usepackage[T1]{fontenc} % ???
\usepackage[francais]{babel} % Typographie française

\usepackage{cleveref}

% Images
\usepackage{graphicx}

% Mise en page
\usepackage[top=2.5cm, bottom=7.5cm, left=2.5cm, right=2.5cm, footskip=4.5cm]{geometry}

% Outils pour l’écriture
\usepackage{xcolor} % Utilisé pour les hiddeux \todo
\usepackage{layout} % Pour voir les commandes utiles

% Couleurs
\usepackage{color}
\definecolor{coloration_numero_ligne}{gray}{0.6}
\definecolor{coloration_reglette}{gray}{0.5}
\definecolor{coloration_fond}{rgb}{1, 1, 1}
\definecolor{coloration_commentaire}{rgb}{0, 0, 0.9}
\definecolor{coloration_mot_cle}{rgb}{0.75, 0, 0}
\definecolor{coloration_chaine}{rgb}{1, 0.2, 1}
\definecolor{coloration_type}{rgb}{0, 0.7, 0}
\definecolor{coloration_erreurs}{rgb}{0.6, 0.35, 0.8}

% Tabulations verbatim
% http://www.grappa.univ-lille3.fr/FAQ-LaTeX/6.16.html
\makeatletter
{\catcode`\^^I=\active
\gdef\verbatim{\catcode`\^^I=\active\def^^I{\hspace*{4em}}%
\@verbatim \frenchspacing\@vobeyspaces \@xverbatim}}
\makeatother

% Coloration syntaxique
\usepackage{minted}
\newminted[shellcode]{shell}{tabsize=4,xleftmargin=1cm,fontsize=\footnotesize,frame=leftline}
% On trouvera probablement difficilement mieux…
\newminted[kamcf]{ini}{tabsize=4,xleftmargin=1cm,fontsize=\footnotesize,frame=leftline}

%%%%%%%%%% Commandes supplémentaires %%%%%%%%%%%

% Marquage du travail restant
% [#1] Texte à afficher
\newcommand{\todo}[1][Il y a là encore des choses à écrire !]{\colorbox{yellow}{#1}}

%%% Gestion du sommaire
% Section non numérotée
% {#1} Nom de la section
\newcommand{\sectionSpeciale}[1]{\section*{#1}\addcontentsline{toc}{section}{#1}}
% Section non numérotée dont le titre n’est pas affiché.
% {#1} Nom de la section
\newcommand{\sectionCachee}[1]{\addcontentsline{toc}{section}{#1}}

%%% Gestion des annexes
\makeatletter
	
	\newcounter{annctr}
	
	% Définir une annexe
	% {#1} Nom de l’annexe
	\newcommand{\annexe}[1]{\stepcounter{annctr}\addcontentsline{ann}{section}{\protect\numberline {\Alph{annctr}}#1}\section*{\numberline {\Alph{annctr}}#1}}
	% Lister les annexes
	% Note : ce serait quand même bien de le faire fonctionner sans avoir besoin du makefile.
	\newcommand\listeannexes{\sectionSpeciale{Annexes}\input{annexes.tex}}
	
	% Déclaration et ouverture du fichier de stockage de la liste des annexes
	\newwrite\tf@ann
	\immediate\openout\tf@ann\jobname.ann\relax
      
\makeatother

%%%%%%%%%% Abréviations %%%%%%%%%%
% Le contenu suffit comme documentation…

\newcommand\my{MySQL}
\newcommand\kam{Kamailio}
\newcommand\rad{RADIUS}
\newcommand\frad{FreeRADIUS}
