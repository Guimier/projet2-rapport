\documentclass[a4paper, oneside, 12pt]{article}

%%%%%%%%%% Paquets %%%%%%%%%%

% Paquets pour le Français
\usepackage[utf8]{inputenc} % Gestion encodages
\usepackage[T1]{fontenc} % ???
\usepackage[francais]{babel} % Typographie française

% Images
\usepackage{graphicx}

% Mise en page
\usepackage[top=2.5cm, bottom=7.5cm, left=2.5cm, right=2.5cm, footskip=4.5cm]{geometry}

% Outils pour l’écriture
\usepackage{xcolor} % Utilisé pour les hiddeux \todo
\usepackage{layout} % Pour voir les commandes utiles

%%%%%%%%%% Commandes supplémentaires %%%%%%%%%%%

% Marquage du travail restant
% [#1] Texte à afficher
\newcommand{\todo}[1][Il y a là encore des choses à écrire !]{\colorbox{yellow}{#1}}

%%% Gestion du sommaire
% Section non numérotée
% {#1} Nom de la section
\newcommand{\sectionSpeciale}[1]{\section*{#1}\addcontentsline{toc}{section}{#1}}
% Section non numérotée dont le titre n’est pas affiché.
% {#1} Nom de la section
\newcommand{\sectionCachee}[1]{\addcontentsline{toc}{section}{#1}}

%%% Gestion des annexes
\makeatletter
	
	% Définir une annexe
	% {#1} Nom de l’annexe
	\newcommand{\annexe}[1]{\newpage\addcontentsline{ann}{section}{#1}\section*{#1}}
	% Lister les annexes
	% Note : ce serait quand même bien de le faire fonctionner sans avoir besoin du makefile.
	\newcommand\listeannexes{\sectionSpeciale{Annexes}\input{annexes.tex}}
	
	% Déclaration et ouverture du fichier de stockage de la liste des annexes
	\newwrite\tf@ann
	\immediate\openout\tf@ann\jobname.ann\relax
      
\makeatother

% Afficher des commandes shell.
\newenvironment{shell}{\todo[Début shell]\\}{\\\todo[Fin shell]}

%%%%%%%%%% Abréviations %%%%%%%%%%
% Le contenu suffit comme documentation…

\newcommand\my{MySQL}
\newcommand\kam{Kamailio}
\newcommand\rad{RADIUS}
\newcommand\frad{FreeRADIUS}