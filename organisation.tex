\section{Organisation du projet}

\subsection{Organisation temporelle}

\begin{figure}[!h]
\begin{center}
\begin{ganttchart}[y unit title=1.5cm,
y unit chart=1.5cm,
vgrid, 
title label anchor/.style={below=-1.6ex},
title left shift=.05,
title right shift=-.05,
title height=1,
bar/.style={fill=blue!25},
bar height=0.7,
bar label node/.append style={align=right}]{0}{23}
%labels\\
\gantttitle{Oct. 2014}{4} 
\gantttitle{Nov. 2014}{4} 
\gantttitle{Déc. 2014}{4} 
\gantttitle{Jan. 2015}{4} 
\gantttitle{Fév. 2015}{4} 
\gantttitle{Mar. 2015}{4} \\
%tasks
\ganttbar{Documentation}{2}{3} \\
\ganttbar{Montage du\ganttalignnewline serveur}{4}{9} \\
\ganttbar{Application de\ganttalignnewline facturation}{10}{17} \\
\ganttbar{Écriture du\ganttalignnewline rapport}{10}{21} \\
\end{ganttchart}
\end{center}
\caption{Diagramme de \nom{Gantt} prévisionnel}
\end{figure}

\begin{figure}[!h]
\begin{center}
\begin{ganttchart}[y unit title=1.5cm,
y unit chart=1.5cm,
vgrid, 
title label anchor/.style={below=-1.6ex},
title left shift=.05,
title right shift=-.05,
title height=1,
bar/.style={fill=blue!25},
bar height=0.7,
bar label node/.append style={align=right}]{0}{23}
%labels\\
\gantttitle{Oct. 2014}{4} 
\gantttitle{Nov. 2014}{4} 
\gantttitle{Déc. 2014}{4} 
\gantttitle{Jan. 2015}{4} 
\gantttitle{Fév. 2015}{4} 
\gantttitle{Mar. 2015}{4} \\
%tasks
\ganttbar{Recherche}{2}{3} \\
\ganttbar{Montage du\ganttalignnewline premier serveur}{4}{9} \\
\ganttbar{Montage du\ganttalignnewline second serveur}{11}{19} \\
\ganttbar{Application\ganttalignnewline de facturation}{11}{17} \\
\ganttbar{Utilisation ATA}{17}{19} \\
\ganttbar{Écriture du\ganttalignnewline rapport}{16}{21} \\
\end{ganttchart}
\end{center}
\caption{Diagramme de \nom{Gantt} réel}
\end{figure}


Dans un premier temps, nous avions prévu de passer les deux premières semaines à nous documenter sur le sujet et les outils à utiliser. Après ce travail de documentation, nous devions entamer le montage du serveur pour lequel nous avions prévu six semaines. Pour finir, on devait alors commencer à coder une application qui facture les conversations téléphoniques passées sur le serveur. Nous avions estimé que cela nous prendrait jusqu'à mi-Février, i.e. environ 8 semaines. Nous devions écrire en parallèle notre rapport. Cela devait durer jusqu'au rendu du rapport en Mars. 

Nous avons commencé notre projet par des recherches sur le proxy SIP {\kam} et le serveur {\frad}. Après les deux semaines prévues, nous avons pu commencé le montage de ces serveurs, sur lequel nous avons passé une grande partie du projet. Durant cette phase, nous avons dû compléter notre travail de documentation afin de pouvoir effectivement installer le serveur.
Nous avons travaillé parallèlement sur deux machines configurées de façons légèrement différentes. Lucien \nom{Guimier} travaillait sur la version disponible dans le dépôt de paquets maintenu par {\kam}, tandis que Damien \nom{Teyssier} travaillait sur une version compilée à partir des sources. Lorsque nous avons réalisé que les paquets proposés n’avaient pas été compilés avec une configuration permettant la communication prévue avec {\frad}, nous nous sommes concentrés sur la version directement compilée à partir des sources, comme l’avaient fait nos prédécesseurs.

Durant le temps de montage du premier serveur, notre travail s’est progressivement organisé et ce manque d’organisation initiale, conjointement avec les difficultés que nous avons rencontrées pour faire communiquer {\kam} et {\frad}, ont causé un manque de notes structurées permettant de monter efficacement le second serveur.


C’est pour cette raison que nous avons ensuite exécuté des tâches en parallèle : Damien \nom{Teyssier} s’est concentré sur l’installation du second serveur, afin de disposer d’une procédure d’installation propre, et Lucien \nom{Guimier} s’est occupé de la configuration des clients matériels et de l’écriture de l’interface de facturation. Cette séparation n’a pas été stricte et nous avons pu à de nombreuses occasions nous aider sur les points difficiles.

L'écriture du rapport a quelque peu pris du retard étant donné que nous ne l'avons commencée qu'au début du mois de Février. Cela est dû aux problèmes que nous avons rencontrés pour monter le second serveur et pour utiliser les {\cph}.


\subsection{Écriture de l’interface de facturation}

Le cahier des charges de notre projet nous imposait de travailler en PHP. Nous avons structuré notre code grâce à la syntaxe orientée objet disponible en PHP 5 et le développement a été réalisé sur plusieurs installations différentes :
\begin{itemize}
	\item notre serveur, qui dispose de la dernière version dans les paquets de la distributions que nous avons utilisée, la 5.4.36 ;
	\item le serveur web de l’ISIMA, \texttt{fc.isima.fr}, qui dispose d’une ancienne version qui n’est plus supportée mais encore appréciée pour sa stabilité, la 5.3.3 ;
	\item l’ordinateur de Lucien \nom{Guimier}, qui dispose de la version 5.5.9.
\end{itemize}

Afin de pouvoir travailler simultanément sur le code, nous avons utilisé le gestionnaire de versions Git, avec un dépôt commun sur Github, disponible à l’adresse \url{https://github.com/Guimier/projet2-php}.

Nous nous sommes efforcés de conserver à jour une documentation de bas niveau, grâce au logiciel Doxygen, qui permet de documenter directement les classes et les méthodes dans les commentaires du code. Cette documentation est générée par la commande \texttt{doxygen} exécutée dans le répertoire racine du dépôt Git. Le code et la documentation sont rédigés en anglais.
