\subsection{\kam}

\subsubsection{Installation}

Pour installer {\kam}, il nous a fallu commencer par installer les dépendances qui n’étaient pas encore présentes dans le système, disponibles dans les paquets :

\begin{itemize}
	\item{gcc} ;
	\item{bison} ;
	\item{libmysqlclient-dev} ;
	\item{libradiusclient-ng-dev} ;
	\item{mysql-server} ;
	\item{mysql-client} ;
	\item{make}.
\end{itemize}

Nous avions initialement envisagé d’utiliser ensuite les paquets Debian disponibles sur les serveurs de {\kam}, mais le paquet de base n’est pas compilé avec le support de {\rad}, rendant le paquet des bibliothèques d’authentification {\rad} inopérant.

Nous avons donc téléchargé et décompressé les sources de {\kam} pour les compiler. Nous avons choisi d’utiliser la dernière version disponible, qui était la version 4.2.2. Pour cela, nous avons utilisé les commandes :

\begin{small}
\begin{verbatim}
wget http://www.kamailio.org/pub/kamailio/4.2.2/src/kamailio-4.2.2_src.tar.gz
tar xvf kamailio-4.2.2_src.tar.gz
\end{verbatim}
\end{small}

Puis nous sommes allés dans le dossier ainsi créé, \texttt{kamailio-4.2.2}, et nous avons exécuté :

\begin{verbatim}
make cfg
\end{verbatim}

Cette commande construit la configuration par défaut pour la compilation, qu’il a fallu adapter à nos besoins. Conformément aux indications du rapport de l’année dernière, nous avons effectué les modifications suivantes :

\begin{itemize}
	\item{dans la liste de modules exclus du fichier \texttt{module.lst}, nous avons retiré les noms des modules concernant {\my} et {\rad}} ;
	\item{dans le fichier \texttt{modules\_k/acc}, nous avons décommenté la ligne suivante :
\begin{verbatim}
ENABLE_RADIUS_ACC=true
\end{verbatim}}
\end{itemize}

Nous avons ensuite compilé et installé {\kam} :

\begin{verbatim}
make all
make install
\end{verbatim}

Sans précision supplémentaire, cette dernière commande installe {\kam} dans \texttt{/usr/local}.

\subsubsection{Configuration}

Nous avons dû ensuite configurer {\kam} dans le répertoire \texttt{/usr/local/etc/kamailio}. Il a fallu rajouter les lignes suivantes dans le fichier de configuration \texttt{kamailio.cfg} :

\begin{verbatim}
#!define WITH_MYSQL
#!define WITH_AUTH
#!define WITH_USRLOCDB
#!define WITH_NAT
\end{verbatim}

Nous avons alors créé puis configuré la base de données. Pour la créer, il a fallu d'abord préciser quel système de base de données nous allions utiliser. Dans notre cas, c'est \my. C'est pourquoi il a fallu aller dans le fichier \texttt{kamctlrc} afin de décommenter la ligne :

\begin{verbatim}
DBENGINE=MYSQL
\end{verbatim}

Ensuite, nous avons tapé 
\begin{verbatim}
/usr/local/sbin/kamdbctl create
\end{verbatim}

Ensuite, il a fallu démarrer le service en tapant :

\begin{verbatim}
kamctl start
\end{verbatim}

Nous avons pu alors ajouter des utilisateurs ainsi :

\begin{verbatim}
kamctl add username password
\end{verbatim}
