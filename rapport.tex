\documentclass[a4paper, oneside, 12pt]{article}

% Paquets pour le Français
\usepackage[utf8]{inputenc} % Gestion encodages
\usepackage[T1]{fontenc} % ???
\usepackage[francais]{babel} % Typographie française

% Images
\usepackage{graphicx}

% Mise en page
\usepackage[top=2cm, bottom=2cm, left=2cm, right=2cm]{geometry}
\usepackage{appendix}
% Outils pour l’écriture
\usepackage{xcolor}
\usepackage{layout}
\usepackage{lipsum}

\newcommand{\todo}[1][Il y a là encore des choses à écrire !]{\colorbox{yellow}{#1}\\}

\newcommand{\sectionSpeciale}[1]{\section*{#1}\addcontentsline{toc}{section}{#1}}
\newcommand{\sectionCachee}[1]{\addcontentsline{toc}{section}{#1}}

%%%%%%%%%%%%%%%%%%%% Gestion des annexes %%%%%%%%%%%%%%%%%%%%
\makeatletter

	\newcommand{\annexe}[1]{\addcontentsline{ann}{section}{#1}\newpage\section*{#1}}
	\newcommand\toann{\sectionSpeciale{Annexes}\input{annexes.tex}}
      \newwrite\tf@ann
      \immediate\openout\tf@ann\jobname.ann\relax
      
\makeatother

%%%%%%%%%%%%%%%%%%%% Abréviations %%%%%%%%%%%%%%%%%%%%

% TODO
\newcommand\bash[1]{\texttt{#1}}

\newcommand\kam{Kamailio}
\newcommand\frad{FreeRADIUS}

%%%%%%%%%%%%%%%%%%%% Début du document %%%%%%%%%%%%%%%%%%%%
\begin{document}

\renewcommand{\contentsname}{Table des matières}
\renewcommand{\listfigurename}{Table des illustrations}
\renewcommand{\figurename}{{\sc Illustration}}

%%%%%%%%%%%%%%%%%%%% Page de garde %%%%%%%%%%%%%%%%%%%%

	% Pas de numéro sur cette page
	\thispagestyle{empty}

	% Note : logo disponibles sur http://fc.isima.fr/~brunot/files/isima/logos/
	% J’avais trouvé une version vectorielle, ce serait mieux, mais je ne la trouve plus…
	\includegraphics[width=6cm]{isima.png}
	
	\vspace{0.5cm}
	
	\begin{minipage}{4cm}
	\begin{flushleft}
		Institut Supérieur d’informatique, de Modélisation et de leurs Applications
		
		\vspace{0.5cm}
		
		\small{ Campus des Cézeaux \\ 24 rue des Landais \\ BP 10125 \\ 63173 Aubière CEDEX }
	\end{flushleft}
	\end{minipage}
	
	\vspace{4cm}

	\begin{center}
		Rapport d’ingénieur \\
		Projet de 2{\ieme} année \\
		Filière {\em{Réseaux et télécommunications}} \\
		\Large{Facturation d’appels sur un serveur Kamailio/Radius}
	\end{center}
	
	\vspace{4cm}
	
	% TODO %

\newpage
	
%%%%%%%%%%%%%%%%%%%% Précontenu %%%%%%%%%%%%%%%%%%%%

\pagenumbering{roman}

%%%%%%%%%% Remerciements %%%%%%%%%%

\sectionSpeciale{Remerciements}

\newpage

%%%%%%%%%% Table des illustrations %%%%%%%%%%

\sectionCachee{\listfigurename}

\listoffigures

\newpage

%%%%%%%%%% Résumé/Abstract %%%%%%%%%%

\sectionSpeciale{Résumé}

\vspace{10cm}

\sectionSpeciale{Abstract}

\newpage

%%%%%%%%%% Table des matiètes %%%%%%%%%%

\sectionCachee{\contentsname}

\tableofcontents

\newpage

%%%%%%%%%%%%%%%%%%%% Contenu %%%%%%%%%%%%%%%%%%%%

% Sauvegarde du numéro de la page courante
\newcounter{metapage}
\setcounter{metapage}{\value{page}}
% Remise à zéro des numéros de page & changement de chiffres
\pagenumbering{arabic}

%%%%%%%%%% Introduction %%%%%%%%%%

\sectionSpeciale{Introduction}

Ceci est une introduction

\newpage

% On verra le placement plus tard…
\section{Installation}
\subsection{Installation de la machine}
Dans le cadre de notre projet nous avons dû installer des machines Linux. Nous avons donc choisi d'installer deux Debian \todo
Ces deux machines devaient nous permettre de monter des serveurs de téléphonie SIP.
C'est pourquoi, nous avons installé un proxy SIP nommé \kam, ainsi qu'un serveur d'authentification \frad.
\subsection{Installation de \kam.}
Ensuite, nous avons installé \kam.
Pour cela, nous avons commencé par installer les paquets suivants :
\begin{itemize}
	\item{gcc}
	\item{bison}
	\item{libmysqlclient-dev}
	\item{libradiusclient-ng-dev}
	\item{mysql-server}
	\item{mysql-client}
	\item{make}
\end{itemize}
Ensuite, il a fallu télécharger les sources de \kam. Pour notre projet, nous avons choisi la version 4.2.2.
Pour cela, nous avons utilisé la commande :

\bash{wget "lien du téléchargement"}

Il a alors fallu extraire les fichiers de l'archive grâce à la commande tar fvx.
Puis nous sommes allés dans le dossier ainsi créé, pour notre part kamailio-4.2.2 et nous avons fait :

\bash{make cfg}

Ensuite, nous sommes allé modifier le fichier module.lst, qui liste les modules exclus.
Nous avons sorti de cette liste, tous les modules de mysql et radius.
Puis nous avons fait : 
	
\bash{make all}


\newpage

%%%%%%%%%% Conclusion %%%%%%%%%%

\sectionSpeciale{Conclusion}

\newpage

%%%%%%%%%%%%%%%%%%%% Postcontenu %%%%%%%%%%%%%%%%%%%%

% Changement des chiffres utilisés pour les numéros de pages
\pagenumbering{roman}
% Rétablissement du numéro de page méta
\setcounter{page}{\value{metapage}}

%%%%%%%%%% Références bibliographiques %%%%%%%%%%

\sectionSpeciale{Références bibliographiques}

\newpage

%%%%%%%%%% Annexes %%%%%%%%%%

\toann

\annexe{Une annexe}
\annexe{Une autre annexe}
\annexe{Encore une annexe}

%%%%%%%%%%%%%%%%%%%% Fin %%%%%%%%%%%%%%%%%%%%

\end{document}
