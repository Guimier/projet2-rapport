\documentclass{article}
\usepackage{pgfgantt}

\begin{document}

\begin{figure}[ftbp]
\begin{center}

\begin{ganttchart}[y unit title=1.5cm,
y unit chart=1cm,
vgrid,hgrid, 
title label anchor/.style={below=-1.6ex},
title left shift=.05,
title right shift=-.05,
title height=1,
bar/.style={fill=gray!50},
incomplete/.style={fill=white},
progress label text={},
bar height=0.7,
group right shift=0,
group top shift=.6,
group height=.3,
group peaks width=.2]{0}{23}
%labels\\
\gantttitle{Oct. 2014}{4} 
\gantttitle{Nov. 2014}{4} 
\gantttitle{Déc. 2014}{4} 
\gantttitle{Jan. 2015}{4} 
\gantttitle{Fév. 2015}{4} 
\gantttitle{Mar. 2015}{4} \\
%tasks
\ganttbar{Recherche}{2}{3} \\
\ganttbar{Montage du premier serveur}{4}{9} \\
\ganttbar{Montage du second serveur}{11}{19} \\
\ganttbar{Application de facturation}{11}{17} \\
\ganttbar{Utilisation ATA}{17}{19} \\
\ganttbar{Écriture du rapport}{16}{21} \\
\end{ganttchart}
\end{center}
\caption{Diagramme de GANTT réel}
\end{figure}

\end{document}