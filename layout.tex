\documentclass[a4paper, oneside, 12pt]{article}

% Paquets pour le Français
\usepackage[utf8]{inputenc} % Gestion encodages
\usepackage[T1]{fontenc} % ???
\usepackage[francais]{babel} % Typographie française

% Images
\usepackage{graphicx}

% Mise en page
\usepackage[top=2cm, bottom=2cm, left=2cm, right=2cm]{geometry}
\usepackage{appendix}
% Outils de mise en page
\usepackage{layout}
\usepackage{lipsum}

\newcommand{\sectionSpeciale}[1]{\section*{#1}\addcontentsline{toc}{section}{#1}}
\newcommand{\sectionCachee}[1]{\addcontentsline{toc}{section}{#1}}
\newcommand{\annexe}[1]{\addcontentsline{ann}{section}{#1}}

\begin{document}

\renewcommand{\contentsname}{Table des matières}
\renewcommand{\listfigurename}{Table des illustrations}
\renewcommand{\figurename}{{\sc Illustration}}

%%%%%%%%%%%%%%%%%%%% Page de garde %%%%%%%%%%%%%%%%%%%%

	% Pas de numéro sur cette page
	\thispagestyle{empty}

	% Note : logo disponibles sur http://fc.isima.fr/~brunot/files/isima/logos/
	% J’avais trouvé une version vectorielle, ce serait mieux, mais je ne la trouve plus…
	\includegraphics[width=6cm]{isima.png}
	
	\vspace{0.5cm}
	
	\begin{minipage}{4cm}
	\begin{flushleft}
		Institut Supérieur d’informatique, de Modélisation et de leurs Applications
		
		\vspace{0.5cm}
		
		\small{ Campus des Cézeaux \\ 24 rue des Landais \\ BP 10125 \\ 63173 Aubière CEDEX }
	\end{flushleft}
	\end{minipage}
	
	\vspace{4cm}

	\begin{center}
		Rapport d’ingénieur \\
		Projet de 2{\ieme} année \\
		Filière {\em{Réseaux et télécommunications}} \\
		\Large{Facturation d’appels sur un serveur Kamailio/Radius}
	\end{center}
	
	\vspace{4cm}
	
	% TODO %

\newpage
	
%%%%%%%%%%%%%%%%%%%% Précontenu %%%%%%%%%%%%%%%%%%%%

\pagenumbering{roman}

%%%%%%%%%% Remerciements %%%%%%%%%%

\sectionSpeciale{Remerciements}

\newpage

%%%%%%%%%% Table des illustrations %%%%%%%%%%

\sectionCachee{\listfigurename}

\listoffigures

\newpage

%%%%%%%%%% Résumé/Abstract %%%%%%%%%%

\sectionSpeciale{Résumé}

\vspace{10cm}

\sectionSpeciale{Abstract}

\newpage

%%%%%%%%%% Table des matiètes %%%%%%%%%%

\sectionCachee{\contentsname}

\tableofcontents

\newpage

%%%%%%%%%%%%%%%%%%%% Contenu %%%%%%%%%%%%%%%%%%%%

% Sauvegarde du numéro de la page courante
\newcounter{metapage}
\setcounter{metapage}{\value{page}}
% Remise à zéro des numéros de page & changement de chiffres
\pagenumbering{arabic}

%%%%%%%%%% Introduction %%%%%%%%%%

\sectionSpeciale{Introduction}

Ceci est une introduction

\newpage

%%%%%%%%%% Partie I %%%%%%%%%%

\section{Première partie}

\subsection{Première sous-partie}

\lipsum[2]

\subsection{Seconde sous-partie}

\lipsum[4]

\newpage

%%%%%%%%%% Partie II %%%%%%%%%%

\section{Seconde partie}

\begin{figure}[h]
	\lipsum[1]
	\caption{Une figure}
\end{figure}

\subsection{Troisième sous-partie}

\lipsum[3]

\subsection{Quatrième sous-partie}

\lipsum[3]

\newpage

%%%%%%%%%% Conclusion %%%%%%%%%%

\sectionSpeciale{Conclusion}

\newpage

%%%%%%%%%%%%%%%%%%%% Postcontenu %%%%%%%%%%%%%%%%%%%%

% Changement des chiffres utilisés pour les numéros de pages
\pagenumbering{roman}
% Rétablissement du numéro de page méta
\setcounter{page}{\value{metapage}}

%%%%%%%%%% Références bibliographiques %%%%%%%%%%

\sectionSpeciale{Références bibliographiques}

\newpage

%%%%%%%%%% Annexes %%%%%%%%%%

\sectionSpeciale{Annexes}

\annexe{Une annexe}

% TODO %

%%%%%%%%%%%%%%%%%%%% Fin %%%%%%%%%%%%%%%%%%%%

\end{document}
