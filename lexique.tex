\sectionSpeciale{Glossaire}

\begin{description}
	
	\item[Balise orpheline] ~ \newline En HTML, la structure du document est constituée de balises, par exemple \texttt{<a>}. Nombre de ces balises définissent des blocs, entre une balise ouvrante et une fermante. Les balises orphelines sont alors les balises ouvrantes ou fermantes sans correspondance.

	\item[Crontab] ~ \newline Fichier de configuration pour les processus récurrents.
	
	\item[Démon]  ~ \newline Processus ou ensemble de processus s’exécutant en arrière-plan. Il s’agit du type de processus utilisé pour créer des services sur un serveur.
	
	\item[Désérialisation] ~ \newline Transformation d'un format de stockage structuré en une structure manipulable.
	
	\item[Open source] ~ \newline Dont le code source est lisible par tous.
	
	\item[Proxy SIP] ~ \newline Serveur servant d’intermédiaire entre les terminaux de communication. En particulier, ce serveur stocke les associations entre les identifiants et les adresses des terminaux. Il sert aussi dans notre cas d’intermédiaire pour la transmission des données, ce qui permet de conserver de traces de l’établissement et de l’arrêt de la communication).
	
	\item[RADIUS] \textit{Remote Authentication Dial-In User Service} \newline Protocole d’authentification, implémenté en particulier par le logiciel \frad.
	
	\item[SIP] \textit{Session Initiation Protocole} \newline Protocole de contrôle des appels en VoIP. C’est le protocole responsable de l’établissement du contact et de son arrêt.
	
	\item[Softphone] ~ \newline Logiciel client de téléphonie.
	
	\item[Timestamp] ~ \newline Le \textit{timestamp} est une valeur permettant la localisation temporelle d’une action. Dans le monde UNIX, il s’agit du nombre de secondes écoulées depuis le 1\ier~janvier 1970.
		
	\item[VoIP] \textit{Voice over Internet Protocol} \newline Ensemble de technologies permettant le transport de la voix sur un réseau utilisant le protocole IP.
\end{description}
