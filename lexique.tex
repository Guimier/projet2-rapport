\sectionSpeciale{Glossaire}

\begin{description}
	\item[Crontab] ~ \newline Fichier de configuration pour les processus récurrents.
	\item[Démon]  ~ \newline Processus ou ensemble de processus s’exécutant en arrière-plan. Il s’agit du type de processus utilisé pour créer des services sur un serveur.
	\item[Proxy SIP] ~ \newline Serveur servant d’intermédiaire entre les terminaux de communication. En particulier, ce serveur stocke les associations entre les identifiants et les adresses des terminaux. Il sert aussi dans notre cas d’intermédiaire pour la transmission des données, ce qui permet de conserver de traces de l’établissement et de l’arrêt de la communication).
	\item[RADIUS] \textit{Remote Authentication Dial-In User Service} \newline Protocole d’authentification, impémenté en particulier par le logiciel \frad.
	\item[SIP] \textit{Session Initiation Protocole} \newline Protocole de contrôle des appels en VoIP. C’est le protocole responsable de l’établissement du contact et son arrêt.
	\item[VoIP] \textit{Voice over Internet Protocol} \newline Ensemble de technologies permettant le transport de la voix sur un réseau utilisant le protocole IP.
\end{description}
