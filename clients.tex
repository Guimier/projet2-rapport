\section{Clients SIP}

Afin de tester le bon fonctionnement des serveurs et de remplir les journaux, nous avons dû utiliser des clients SIP. Nous avons principalement utilisé les clients logiciels, mais nous avons aussi essayé des clients matériels.

\subsection{\xlite}

Suite à la rapide présentation qui nous en avait été faite en cours de téléphonie, nous avons commencé par utiliser {\xlite}. {\xlite} est un client SIP logiciel développé par la société CounterPath. Il est disponible pour {\win} et {\mac}. Le fait qu’il ne soit plus disponible pour {\lnx} nous a amené à peu l’utiliser, afin de simplifier notre organisation.

\todo[Expliquer brièvement la configuration / mettre une image]

\subsection{\lnp}

Nous avons donc recherché un client logiciel disponible pour {\lnx} ; après avoir testé plusieurs logiciels non fonctionnels à cause d’erreurs de compilation, dues à des incompatibilité avec des bibliothèques trop récentes, et d’erreurs de segmentation à l’exécution, nous avons trouvé {\lnp}, qui fonctionne bien et qui a l’avantage d’être disponible dans les dépôts de la plupart des distributions {\lnx}.

\todo[Étoffer un peu la description ?]

{\lnp} est un logiciel \textit{open source} disponible pour les principaux systèmes d’exploitation, aussi bien bureau que mobile.

\todo[Expliquer brièvement la configuration / mettre une image]

\subsection{\cph}

\todo[Éventuellement photo ; problème des mots de passe]

\subsection{\cata}

\todo[Éventuellement photo ; description de la configuration ; problème de la résolution de nom]
