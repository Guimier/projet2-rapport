\section{Installation}
\subsection{Installation de la machine}
Dans le cadre de notre projet nous avons dû installer des machines Linux. Nous avons donc choisi d'installer deux Debian \todo
Ces deux machines devaient nous permettre de monter des serveurs de téléphonie SIP.
C'est pourquoi, nous avons installé un proxy SIP nommé \kam, ainsi qu'un serveur d'authentification \frad.
\subsection{Installation de \kam.}
Ensuite, nous avons installé \kam.
Pour cela, nous avons commencé par installer les paquets suivants :
\begin{itemize}
	\item{gcc}
	\item{bison}
	\item{libmysqlclient-dev}
	\item{libradiusclient-ng-dev}
	\item{mysql-server}
	\item{mysql-client}
	\item{make}
\end{itemize}
Ensuite, il a fallu télécharger les sources de \kam. Pour notre projet, nous avons choisi la version 4.2.2.
Pour cela, nous avons utilisé la commande :

\bash{wget "lien du téléchargement"}

Il a alors fallu extraire les fichiers de l'archive grâce à la commande tar fvx.
Puis nous sommes allés dans le dossier ainsi créé, pour notre part kamailio-4.2.2 et nous avons fait :

\bash{make cfg}

Ensuite, nous sommes allé modifier le fichier module.lst, qui liste les modules exclus.
Nous avons sorti de cette liste, tous les modules de mysql et radius.
Puis nous avons fait : 
	
\bash{make all}
