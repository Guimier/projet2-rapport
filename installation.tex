\section{Installation d’un serveur}

\subsection{Installation du système}

La première étape de notre projet consistait en l’installation de serveurs de téléphonie SIP. Nous en avons installé deux, fontionnant avec le système d’exploitation Debian. Nous avons choisi d’installer la dernière version stable disponible au moment de l’installation, la 7.7.0 (\textit{Wheezy}) en version \textit{i386}, à partir de l’installeur \textit{netinstall}, c’est-à-dire celui qui profite du réseau pour télécharger les paquets qui ne sont pas indispensables, afin de réduire la quantité d’information à graver sur le CD-ROM.

Conformément à notre cahier des charges, nous y avons installé un proxy SIP, {\kam}, attaché à un serveur d’authentification {\rad}, {\frad}.

Dans la suite, les chemins relatifs indiqués dans les descriptions de la configuration sont relatifs au répertoire principal de configuration du logiciel concerné.

\subsection{\kam}

\subsubsection{Installation}

Pour installer {\kam}, il nous a fallu commencer par installer les dépendances qui n’étaient pas encore présentes dans le système, disponibles dans les paquets :

\begin{itemize}
	\item{gcc} ;
	\item{bison} ;
	\item{libmysqlclient-dev} ;
	\item{libradiusclient-ng-dev} ;
	\item{mysql-server} ;
	\item{mysql-client} ;
	\item{make}.
\end{itemize}

Nous avions initialement envisagé d’utiliser ensuite les paquets Debian disponibles sur les serveurs de {\kam}, mais le paquet de base n’est pas compilé avec le support de {\rad}, rendant le paquet des bibliothèques d’authentification {\rad} inopérant.

Nous avons donc téléchargé et décompressé les sources de {\kam} pour les compiler. Nous avons choisi d’utiliser la dernière version disponible, qui était la version 4.2.2. Pour cela, nous avons utilisé les commandes :

\begin{small}
\begin{verbatim}
wget http://www.kamailio.org/pub/kamailio/4.2.2/src/kamailio-4.2.2_src.tar.gz
tar xvf kamailio-4.2.2_src.tar.gz
\end{verbatim}
\end{small}

Puis nous sommes allés dans le dossier ainsi créé, \texttt{kamailio-4.2.2}, et nous avons exécuté :

\begin{verbatim}
make cfg
\end{verbatim}

Cette commande construit la configuration par défaut pour la compilation, qu’il a fallu adapter à nos besoins. Conformément aux indications du rapport de l’année dernière, nous avons effectué les modifications suivantes :

\begin{itemize}
	\item{dans la liste de modules exclus du fichier \texttt{module.lst}, nous avons retiré les noms des modules concernant {\my} et {\rad}} ;
	\item{dans le fichier \texttt{modules\_k/acc}, nous avons décommenté la ligne suivante :
\begin{verbatim}
ENABLE_RADIUS_ACC=true
\end{verbatim}}
\end{itemize}

Nous avons ensuite compilé et installé {\kam} :

\begin{verbatim}
make all
make install
\end{verbatim}

Sans précision supplémentaire, cette dernière commande installe {\kam} dans \texttt{/usr/local}.

\subsubsection{Configuration}

Nous avons dû ensuite configurer {\kam} dans le répertoire \texttt{/usr/local/etc/kamailio}. Il a fallu rajouter les lignes suivantes dans le fichier de configuration \texttt{kamailio.cfg} :

\begin{verbatim}
#!define WITH_MYSQL
#!define WITH_AUTH
#!define WITH_USRLOCDB
#!define WITH_NAT
\end{verbatim}

Nous avons alors créé puis configuré la base de données. Pour la créer, il a fallu d'abord préciser quel système de base de données nous allions utiliser. Dans notre cas, c'est \my. C'est pourquoi il a fallu aller dans le fichier \texttt{kamctlrc} afin de décommenter la ligne :

\begin{verbatim}
DBENGINE=MYSQL
\end{verbatim}

Ensuite, nous avons tapé 
\begin{verbatim}
/usr/local/sbin/kamdbctl create
\end{verbatim}

Ensuite, il a fallu démarrer le service en tapant :

\begin{verbatim}
kamctl start
\end{verbatim}

Nous avons pu alors ajouter des utilisateurs ainsi :

\begin{verbatim}
kamctl add username password
\end{verbatim}

\subsection{\frad}
\subsubsection{Installation}

Nous avons installé {\frad} ainsi que les outils de test associés à partir des paquets disponibles. Pour cela, il a fallu taper la commande :

\begin{verbatim}
apt-get install freeradius freeradius-utils
\end{verbatim}

Puis, pour finir, nous avons installé le client {\rad} proposé par {\frad} grâce à la commande suivante :

\begin{verbatim}
apt-get install radiusclient1
\end{verbatim}

\subsubsection{Configuration}
Nous avions fini d'installer {\frad}. Il nous restait donc à le configurer. Ce serveur {\rad} peut être configuré de deux manières différentes : soit à partir des fichiers présents dans \texttt{/etc/freeradius}, soit à partir d'une base de données. Nous avons choisi de configurer à partir des fichiers. Il a donc fallu aller vérifier si la configuration se faisait bien grâce aux fichiers. Pour cela, il a suffi d'aller vérifier dans le fichier \texttt{sites-available/default}.


Nous avons dû ensuite modifier le fichier \texttt{/etc/freeradius/users}. Nous avons tout effacé et avons annoncé les utilisateurs autorisés ou non de la manière suivante :

\begin{verbatim}
7000@192.168.102.120 Auth-Type:=Accept
7001@192.168.102.120 Auth-Type:=Accept
7003@192.168.102.120 Auth-Type:=Reject
\end{verbatim}

Cela autorise les utilisateurs 7000 et 7001 mais rejette l'utilisateur 7002 dans le domaine 192.168.102.120.

Ainsi nous avons deux sécurités à passer afin de pouvoir utiliser le serveur {\kam}. Il faut tout d'abord être dans la base de donnée de {\kam} puis ensuite dans le fichier users de {\frad} en tant qu'utilisateur accepté. Nous avons pu alors vérifier :

\begin{verbatim}
service freeradius reload
radtest login password localhost 0 testing123
\end{verbatim}

La première commande permet de forcer {\frad} à relire ses fichiers de configuration. La second retourne des informations sur sa tentative de connexion au serveur {\frad}, terminant par le status \texttt{Access-Accept} si tout s’est bien déroulé.

Le paramètre \texttt{testing123} est le mot de passe donné par défaut au client localhost dans le fichier \texttt{client.conf}. Nous l'avons par la suite changé par \texttt{kamailio}. Le champ \texttt{password} n’importe pas ici, puisque nous laissons la vérification du mot de passe à {\kam}.

\subsubsection{Intégration de {\rad} à {\kam}}

Pour commencer cette partie, il a fallu copier les deux  dictionnaires {\rad} \texttt{dictionary.kamailio} et \texttt{dictionary.sip} dans le répertoire du client {\rad} \texttt{/etc/radiusclient-ng} :

\begin{small}
\begin{verbatim}
cp /usr/local/etc/kamailio/dictionary.kamailio /etc/radiusclient-ng
cp /usr/local/etc/kamailio/dictionary.sip /etc/radiusclient-ng
\end{verbatim}
\end{small}

Puis, nous les avons inclus dans \texttt{/etc/radiusclient-ng/dictionary} :

\begin{verbatim}
$INCLUDE dictionary.kamailio
$INCLUDE dictionary.sip
\end{verbatim}

Puis il a fallu décommenter les lignes des attributs ou des valeurs \texttt{Failed}, \texttt{Sip-Session} et \texttt{Call-Check} dans les trois fichiers de dictionnaire {\rad}.

Nous avons ensuite dû vérifier que le fichier de configuration du client {\rad}, \texttt{/etc/radiusclient-ng/radiusclient.conf}, contient les deux lignes suivantes, qui donnent l'adresse du serveur {\rad} :

\begin{verbatim}
authserver localhost
acctserver localhost
\end{verbatim}

Puis, dans le but d'annoncer notre serveur, nous avons dû modifier le fichier \texttt{/etc/radiusclient-ng/servers} en rajoutant la ligne suivante :

\begin{verbatim}
192.168.102.120		kamailio
\end{verbatim}

Ce fichier contient l'adresse des serveurs {\rad} et leurs clefs. Ainsi, nous avons choisi la clef \texttt{kamailio}.
Puis ensuite il a fallu aller modifier le fichier \texttt{/etc/freeradius/clients.conf} qui définit les clients autorisés par le serveur {\frad} :

\begin{verbatim}
Client localhost
{
 ipaddress = 192.168.102.120
 secret = kamailio
}
\end{verbatim}

Ensuite, nous avons dû modifier le fichier de configuration de {\kam}, \texttt{/usr/local/kamailio/kamailio.cfg}. Nous avons ajouté les lignes permettants de charger les modules nécessaires au bon fonctionnement de {\kam} avec {\frad} :

\begin{verbatim}
#!PRADIUS
loadmodule "acc_radius.so"
loadmodule "auth_radius.so"
loadmodule "misc_radius.so"

modparam("acc","log_level", 1)

modparam("acc","radius_flag", 1)
modparam( "acc", "radius_missed_flag", 2 )
modparam( "acc", "log_flag", 1 )
modparam( "acc", "log_missed_flag", 1 )
modparam( "acc", "service_type", 15 )
modparam( "acc|auth_radius|misc_radius", "radius_config",
	"/etc/radiusclient-ng/radiusclient.conf" )

modparam( "auth_radius", "service_type", 15 )

modparam( "acc|acc_radius", "radius_extra",
	"Calling-Station-Id=$fu; Called-Station-Id=$tu" )

#!endif
\end{verbatim}

\label{extension}
Le paramètre \texttt{radius\_extra} permet d’ajouter des champs dans les enregistrement du journal de {\frad} [3]. Nous y ajoutons ici les identifiants des deux terminiaux communicant [2].

Il a, enfin, fallu ajouter la demande d'authentification radius dans la logique de routage situé dans le même fichier :

\begin{verbatim}
#!ifdef PRADIUS
        if (is_method("REGISTER"))
        {
                if (!radius_www_authorize("$fd"))
                {
                        www_challenge("$fd","0");
                        exit;
                }
        }
#!endif
\end{verbatim}




\subsection{\apa}

Pour travailler sur l’interface de facturation web en PHP, nous avons installé {\apa} :

\begin{verbatim}
apt-get install apache2 libapache2-mod-php5
\end{verbatim}

La configuration par défaut d’{\apa} est suffisante pour notre travail : la racine web est définie comme le répertoire \texttt{/var/www}, dans lequel nous avons ajouté un répertoire \texttt{logs} qui contenait notre code. Nous n’avons pas travaillé directement à la racine afin de pouvoir utiliser {\pma} dans un autre répertoire.

Nous pouvions donc accéder à notre interface web \textit{via} l’adresse
\begin{verbatim}
http://localhost/logs/
\end{verbatim}

