\section{Installation}

\subsection{Installation du système}

\todo[Trouver la version de Debian]

La première étape de notre projet consistait en l’installation de serveurs de téléphonie SIP. Nous en avons installé deux, fontionnant avec le système d’exploitation Debian. Conformément à notre cahier des charges, nous y avons installé un proxy SIP, {\kam}, attaché à un serveur d’authentification {\rad}, {\frad}.

\subsection{Installation de {\kam}}

	Pour installer {\kam}, il nous a fallu commencer par installer les dépendances qui n’étaient pas encore présentes dans le système, disponibles dans les paquets :
	
\begin{itemize}
	\item{gcc} ;
	\item{bison} ;
	\item{libmysqlclient-dev} ;
	\item{libradiusclient-ng-dev} ;
	\item{mysql-server} ;
	\item{mysql-client} ;
	\item{make}.
\end{itemize}

Nous avions initialement envisagé d’utiliser ensuite les paquets Debian disponibles sur les serveurs de {\kam}, mais le paquet de base n’est pas compilé avec le support de {\rad}, rendant le paquet des bibliothèques d’authentification {\rad} inopérant.

Nous avons donc téléchargé et décompressé les sources de {\kam} pour les compiler. Nous avons choisi d’utiliser la dernière version disponible, qui était la version 4.2.2. Pour cela, nous avons utilisé les commandes :

\begin{shell}
wget http://www.kamailio.org/pub/kamailio/4.2.2/src/kamailio-4.2.2\_src.tar.gz
tar xvf kamailio-4.2.2\_src.tar.gz
\end{shell}

Puis nous sommes allés dans le dossier ainsi créé, pour notre part kamailio-4.2.2 et nous avons exécuté :

\begin{shell}
make cfg
\end{shell}

Cette commande construit la configuration par défaut pour la compilation, qu’il a fallu adapter à nos besoins. Conformément aux indications du rapport de l’année dernière, nous avons effectué les modifications suivantes :

\begin{itemize}
	\item{dans la liste de modules exclus du fichier \texttt{module.lst}, nous avons retiré les noms des modules concernant {\my} et {\rad}}.
\end{itemize}

Nuos avons ensuite compilé et installé {\kam} :
	
\begin{shell}
make all
make install
\end{shell}

Cette dernière commande installe {\kam} dans \texttt{/usr/local}.
