\section{Installation}
\subsection{Installation de la machine}
Dans le cadre de notre projet nous avons dû installer des machines Linux. Nous avons donc choisi d'installer deux Debian \todo
Ces deux machines devaient nous permettre de monter des serveurs de téléphonie SIP.
C'est pourquoi, nous avons installé un proxy SIP nommé \kam, ainsi qu'un serveur d'authentification \frad.
\subsection{\kam.}
\subsubsection{Installation}
Ensuite, nous avons installé \kam.
Pour cela, nous avons commencé par installer les paquets suivants :
\begin{itemize}
	\item{gcc}
	\item{bison}
	\item{libmysqlclient-dev}
	\item{libradiusclient-ng-dev}
	\item{mysql-server}
	\item{mysql-client}
	\item{make}
\end{itemize}
Ensuite, il a fallu télécharger les sources de \kam. Pour notre projet, nous avons choisi la version 4.2.2.
Pour cela, nous avons utilisé la commande :

\bash{wget "lien du téléchargement"}

Il a alors fallu extraire les fichiers de l'archive grâce à la commande tar fvx.
Puis nous sommes allés dans le dossier ainsi créé, pour notre part kamailio-4.2.2 et nous avons fait :

\bash{make cfg}

Ensuite, nous sommes allés modifier le fichier module.lst, qui liste les modules exclus.
Nous avons sorti de cette liste, tous les modules de mysql et radius.
Puis nous avons fait : 
	
\bash{make all}

Puis :

\bash{make install}

\subsubsection{Configuration}

Nous avons dû ensuite configurer \kam. Nous sommes allés dans le dossier /usr/local/etc/kamailio. Il a fallu rajouter les lignes suivantes dans le fichier de configuration kamailio.cfg :

\#!define WITH\_MYSQL

\#!define WITH\_AUTH

\#!define WITH\_USRLOCDB

\#!define WITH\_NAT

Nous avons alors créé puis configuré la base de données. Pour la créer, il a fallu d'abord préciser quel système de base de données nous allions utiliser.
Dans notre cas, c'est mysql. C'est pourquoi il a fallu aller dans le fichier /usr/local/etc/kamailio/kamctlrc afin de décommenter la ligne :

DBENGINE=MYSQL

Ensuite, nous avons tapé \bash{/usr/local/sbin/kamdbctl create}
\todo

\subsection{\frad}
\subsubsection{Installation}

Nous avons choisi ensuite d'installer \frad à partir des paquets présents sur la machine. Pour cela, il a fallu taper la commande :

\bash{apt-get install freeradius}

Il fallait aussi installer les outils nécessaires au bon fonctionnement de \frad :

\bash{apt-get install freeradius-utils}

Puis, pour finir, nous avons installé un client Radius nommé Radiusclient grâce à la commande suivante :

\bash{apt-get install radiusclient1}

\subsubsection{Configuration}
Nous avions fini d'installer \frad. Il nous restait donc à le configurer. Nous somme donc allés dans /etc/freeradius afin de modifier le fichier users. Dans ce fichier, nous avons ajouté des utilisateurs de la façon suivantes :

login Cleartext-Password:="password"

Cette ligne autorise l'utilisateur login à se connecter avec le mot de passe "password".






