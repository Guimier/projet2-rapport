\section{Installation}

\subsection{Installation du système}

\todo[Trouver la version de Debian]

La première étape de notre projet consistait en l’installation de serveurs de téléphonie SIP. Nous en avons installé deux, fontionnant avec le système d’exploitation Debian. Conformément à notre cahier des charges, nous y avons installé un proxy SIP, {\kam}, attaché à un serveur d’authentification {\rad}, {\frad}.

\subsection{\kam}

\subsubsection{Installation}

Pour installer {\kam}, il nous a fallu commencer par installer les dépendances qui n’étaient pas encore présentes dans le système, disponibles dans les paquets :

\begin{itemize}
	\item{gcc} ;
	\item{bison} ;
	\item{libmysqlclient-dev} ;
	\item{libradiusclient-ng-dev} ;
	\item{mysql-server} ;
	\item{mysql-client} ;
	\item{make}.
\end{itemize}

Nous avions initialement envisagé d’utiliser ensuite les paquets Debian disponibles sur les serveurs de {\kam}, mais le paquet de base n’est pas compilé avec le support de {\rad}, rendant le paquet des bibliothèques d’authentification {\rad} inopérant.

Nous avons donc téléchargé et décompressé les sources de {\kam} pour les compiler. Nous avons choisi d’utiliser la dernière version disponible, qui était la version 4.2.2. Pour cela, nous avons utilisé les commandes :

\begin{shellcode}
wget http://www.kamailio.org/pub/kamailio/4.2.2/src/kamailio-4.2.2_src.tar.gz
tar xvf kamailio-4.2.2_src.tar.gz
\end{shellcode}

Puis nous sommes allés dans le dossier ainsi créé, \texttt{kamailio-4.2.2}, et nous avons exécuté :

\begin{shellcode}
make cfg
\end{shellcode}

Cette commande construit la configuration par défaut pour la compilation, qu’il a fallu adapter à nos besoins. Conformément aux indications du rapport de l’année dernière, nous avons effectué les modifications suivantes :

\begin{itemize}
	\item{dans la liste de modules exclus du fichier \texttt{module.lst}, nous avons retiré les noms des modules concernant {\my} et {\rad}} ;
	\item{dans le fichier \texttt{modules\_k/acc}, nous avons décommenté la ligne \kamcfil{ENABLE\_RADIUS\_ACC=true}}.
\end{itemize}

Nous avons ensuite compilé et installé {\kam} :

\begin{shellcode}
make all
make install
\end{shellcode}

Cette dernière commande installe {\kam} dans \texttt{/usr/local}.

\subsubsection{Configuration}

Nous avons dû ensuite configurer {\kam}. Nous sommes allés dans le dossier \texttt{/usr/local/etc/kamailio}. Il a fallu rajouter les lignes suivantes dans le fichier de configuration \texttt{kamailio.cfg} :

\begin{kamcf}
#!define WITH_MYSQL
#!define WITH_AUTH
#!define WITH_USRLOCDB
#!define WITH_NAT
\end{kamcf}

Nous avons alors créé puis configuré la base de données. Pour la créer, il a fallu d'abord préciser quel système de base de données nous allions utiliser.
Dans notre cas, c'est mysql. C'est pourquoi il a fallu aller dans le fichier \texttt{/usr/local/etc/kamailio/kamctlrc} afin de décommenter la ligne :

\begin{kamcf}
DBENGINE=MYSQL
\end{kamcf}

Ensuite, nous avons tapé 
\begin{shellcode}
/usr/local/sbin/kamdbctl create
\end{shellcode}
\todo

\subsection{\frad}
\subsubsection{Installation}

Nous avons installé {\frad} ainsi que les outils de test associés à partir des paquets disponibles. Pour cela, il a fallu taper la commande :

\begin{shellcode}
apt-get install freeradius freeradius-utils
\end{shellcode}

Puis, pour finir, nous avons installé le client {\rad} proposé par {\frad} grâce à la commande suivante :

\begin{shellcode}
apt-get install radiusclient1
\end{shellcode}

\subsubsection{Configuration}
Nous avions fini d'installer {\frad}. Il nous restait donc à le configurer. Ce serveur {\rad} peut être configuré de deux manières différentes : soit à partir des fichiers présents dans \texttt{/etc/freeradius}, soit à partir d'une base de données. Nous avons choisi de configurer à partir des fichiers. Il a donc fallu aller vérifier si la configuration se faisait bien grâce aux fichiers. Pour cela, il a suffi d'aller vérifier dans le fichier \texttt{/etc/freeradius/sites-available/default}.

Ensuite, nous sommes allés dans le répertoire \texttt{/etc/freeradius} afin de modifier le fichier \texttt{users}. Dans ce fichier, nous avons ajouté des utilisateurs de la façon suivantes :

login Cleartext-Password:="password"

Cette ligne autorise l'utilisateur login à se connecter avec le mot de passe "password".

Nous avons pû ensuite faire le test en utilisant la commande suivante :

\begin{shellcode}
radtest login password localhost 0 testing123
\end{shellcode}

Le paramètre "testing123" est le mot de passe donné par défaut au client localhost dans le fichier \texttt{/etc/freeradius/client.conf}. Nous l'avons ensuite changé en "kamailio".

\subsubsection{Intégration de {\rad} à {\kam}}

Pour commencer, il a fallu copier les dictionnaires \texttt{dictionary.kamailio} et \texttt{dictionary.sip} dans le répertoire \texttt{/etc/radiusclient-ng} :

\begin{shellcode}
cp /usr/local/etc/kamailio/dictionary.kamailio /etc/radiusclient-ng
cp /usr/local/etc/kamailio/dictionary.sip /etc/radiusclient-ng
\end{shellcode}

Puis, nous avons inclus dans \texttt{/etc/radiusclient-ng/dictionary} les deux fichiers suivants :

\begin{itemize}
\item{\texttt{/etc/radiusclient-ng/dictionary.kamailio} ;}
\item{\texttt{/etc/radiusclient-ng/dictionary.sip}.}
\end{itemize}

Puis il a fallu décommenter les lignes des attributs ou des valeurs Failed, Sip-Session et Call-Check dans les trois fichiers dictionary.

Nous avons ensuite dû vérifier que les deux lignes suivantes,qui donnent l'adresse du serveur {\rad}, se trouvaient bien dans le fichier \texttt{/etc/radiusclient-ng/radiusclient.conf} :

\begin{verbatim}
authserver localhost
acctserver localhost
\end{verbatim}

