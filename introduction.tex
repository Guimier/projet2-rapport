\sectionSpeciale{Introduction}

La VoIP est devenue l'avenir de nos conversations téléphoniques. En effet, la téléphonie commutée (dite classique) est vouée à disparaitre dans un avenir proche. Dans le milieu des entreprises, la VoIP est aujourd'hui majoritaire même si la transition fût difficile due au coût qu'elle demandait aux entreprises. Cependant, cette solution a un avantage de coût à long terme puisque qu'elle permet de faire passer toutes les communications sur le réseau de transfert de données et ainsi réduire les coûts de communication.

C'est pour cette raison que l'on nous a demandé de continuer le projet de madame Salimata \nom{Ndiaye} et monsieur Quentin \nom{Volant}. On a donc dû suivre leur protocole afin de pouvoir remonter un serveur de téléphonie en VoIP {\kam} et un serveur d'authentification {\rad}.

Nous devions ainsi réécrire une procédure d'installation de ces serveurs et, ensuite, coder une application de facturation des communications VoIP pouvant être utilisée par une entreprise.

Dans un premier temps, nous allons voir notre protocole d'installation des serveurs {\kam} et {\frad}. Nous allons ensuite expliquer les différents outils de communication que nous avons essayés d'utilisés sur notre serveur. Finalement, nous présenterons le fonctionnement de l'application de facturation.